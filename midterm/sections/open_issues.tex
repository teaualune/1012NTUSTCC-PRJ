\subsection{Open Issues}

One of the severe problem in distributed computing is the Internet latency. While distributed computing aims to speed up computation by using multiple machines as computing resources, the communicaiton would cost a lot if the problem is not dealed carefully.

For the original MapReduce model, the computation takes place on a cluster of computers that are connected closely. But our system uses web browser as computation sources, thus performance would be severely reduced by latencies between server and browsers.

The second possible bottleneck is the execution speed of javascript codes. Javascript is usually executed in a built environment that can be abstractly described as JSVM; famous examples include Google V8 engine, Mozilla Rhino, etc.
Javascript runs dynamically with an interpreter and does not perform well comparing to compiled programming langauges; moreover, most of the javascript engines in browsers have only limited compute ability and might slow down the overall performance.

Another issue is the performance evaluatation and demonstration part: it is difficult to find a suitable benchmark or test cases that can fairly evaluate our system. Finding lots of browsers to do a demonstration is another hard task.