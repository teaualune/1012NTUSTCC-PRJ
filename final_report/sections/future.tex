\section{Future Works}

Fault tolerance is the biggest unsolved issue in the project. MapReduce needs a lot of data transmitions. If one connection node failed, the whole algorithm may break.
What's worse, it is normal for a client disconnects from the server, so fault tolerance should be carefully considered; but now we are unable to come up a good solution for this issue.

For benchmark and evaluation, our model is hard to evaluate and do performance comparison. Our system needs different benchmarks from traditional MapReduce model, and is unsuitable to compare with traditional models.

Peer-to-peer (P2P) communication is needed to transmit the intermediate data between mappers and reducers. WebRTC\cite{webRTC} will be implemented by most of the browsers in the future, which is a solution to enable browser-to-browser data exchange.

In the original idea the mappers and reducers are excuted via WebWorker\cite{webworker}, which provides multi-threading javascript execution functionality in browsers. If mapper and reducer tasks are heavy, they can be executed on the worker thread, but the data synchronization issue between main thread and worker thread is hard to solve so currently it is not implemented in the project.
