\section{Motivation}

MapReduce\cite{mapreduce} provides a distributed way for large-scale data processing. In the traditional MapReduce model, a server node dispatches data and tasks to various client nodes and collect results after mapper and reducer functions are done by clients. Both the server and clients need efforts to configure on connection and execution environment for MapReduce programs.

In order to get many clients in a more convenient way, we introduce a new kind of client, i.e. browsers.
In the initial scheme of our idea, researchers who need more computing resources do not need to set up lots of clients; instead, they can use a HTTP server that establishes connection with clients who visit the website (server).
Data, mappers and reducers can therefore dispatch to those browsers and execute.
To support a research, a resource provider only needs to keep a browser open and connected to the server.

Due to the lack of cross-browser communication mechanism in the current HTML standard, the intermediate data have to send back to the server and then dispatch to clients again. This will become a bottleneck.

And because the server and browsers communicate via HTTP protocol, the interaction between the server and clients is much slower than in clusters.
Also the browser is not a fast execution environment. These challenges should be considerd during implementation.

We can extend the functionality by enabling other Node.js applications to become clients; thus researchers who are doubt for the speed of browser computing can set up client machines with node.js and libraries we provided.